% You should title the file with a .tex extension (hw1.tex, for example)
\documentclass[11pt]{article}

\usepackage{amsmath}
\usepackage{mathtools}
\usepackage{amssymb}
\usepackage{wrapfig}
\usepackage{fancyhdr}
\usepackage{tikz-qtree}
\usepackage{tikz-qtree-compat}
\usepackage[normalem]{ulem}
\usepackage{tikz}
\usepackage{graphicx}
\DeclareMathOperator*{\argmin}{argmin}
\DeclareMathOperator*{\argmax}{argmax}
\usepackage{hyperref}
\usepackage{soul}
\usepackage{natbib}

\oddsidemargin0cm
\topmargin-2cm     %I recommend adding these three lines to increase the 
\textwidth16.5cm   %amount of usable space on the page (and save trees)
\textheight23.5cm  

\newcommand{\question}[2] {\vspace{.25in} \hrule\vspace{0.5em}
\noindent{\bf #1: #2} \vspace{0.5em}
\hrule \vspace{.10in}}
\renewcommand{\part}[1] {\vspace{.10in} {\bf (#1)}}

\newcommand{\myname}{Sean Bittner}
\newcommand{\myandrew}{srb2201@columbia.edu}
\newcommand{\myhwnum}{12}

\setlength{\parindent}{0pt}
\setlength{\parskip}{5pt plus 1pt}
\setlength{\bibsep}{0pt plus 0.3ex}

 
\DeclarePairedDelimiter\abs{\lvert}{\rvert}%
 
\pagestyle{fancyplain}
\rhead{\fancyplain{}{\myname\\ \myandrew}}

\begin{document}

\medskip                        % Skip a "medium" amount of space
                                % (latex determines what medium is)
                                % Also try: \bigskip, \littleskip

\thispagestyle{plain}
\begin{center}                  % Center the following lines
{\Large Thesis committee report} \\
Sean Bittner, January 16, 2021
\end{center}

\textbf{Summary}:
During my last thesis committee meeting in May, 2020, I presented progress on the emergent property inference (EPI) project.  
To refresh, EPI is a likelihood-free inference (LFI) method, that uses deep probability distributions (normalizing flows) to obtain flexible posterior approximations.  
I presented the EPI method and my work on neuroscientific applications of EPI, which come from this manuscript \cite{bittner2019interrogating}.
Our conclusion was that I should focus on a.) improving the science done with this method and b.) providing a comparative analysis between EPI and other LFI techniques.
The advancements I've made since our last meeting are listed below.

\underline{Advancements since May, 2020}:
\begin{enumerate}
\item \textbf{STG}: This introductory example based on an STG subcircuit \cite{gutierrez2013multiple} has been revisited: we introduced stochasticity to make a technical point, and changed the emergent property to improve the pedagogical value of the analysis.
\item \textbf{V1}: We now focus on how parameters of noise govern variability in the stochastic stabilized supralinear network (SSSN) \cite{hennequin2018dynamical} generalized to have inhibitory multiplicity.
\item \textbf{SC}: By request from our collaborators, we changed the modeling of network responses from winner-take-all to comparative decision making.
Intriguingly, EPI inferred connectivities postdict optogenetic response properties published by the Brody Lab.
\item \textbf{RNNs}: We infer connectivities of low-rank RNN's exhibiting stable amplification using criteria defined in \cite{bondanelli2020coding}.  
Here, we demonstrate the superior scaling properties of EPI compared to sequential Monte Carlo approximate Bayesian computation (SMC-ABC) \cite{sisson2007sequential} and sequential neural posterior estimation (SNPE) \cite{gonccalves2020training}.
\end{enumerate}

Each model in the paper was re-analyzed using the new {\color{blue} \href{https://github.com/cunningham-lab/epi}{epi} software package}, which takes advantage of modern advancements in normalizing flows.

In September, I had the opportunity to present our work on EPI at the Bernstein Conference Workshop on ``Inferring and Testing Optimality in Perception and Neurons".
Upon feedback from the committee and some final touches, we will resubmit our manuscript to eLife.

\textbf{Professional ambitions}:
This past summer (June-Aug 2020) I had a research internship at Facebook Reality Labs (with the CTRL-Labs Science Team).
I used recently developed techniques from automatic speech recognition to substantially improve EMG typing accuracy.
I'm excited about the challenging research problems in BCI development, and I've applied for a Research Scientist position at Facebook with the CTRL-Labs group.
I will be interviewed soon, the start date would be flexible, and my preferred graduation date is still May 2021.
Here is my proposal for the thesis outline.

\textbf{Proposed thesis outline} \\
\underline{Title}: Bridging data, theory, and experiment in computational neuroscience. \\
\underline{Chapter 1}: Understanding movement generation using normative models: \\ 
$~~~~~~~~~~~~~~~~~~~$ tangling \cite{russo2018motor} and divergence \cite{russo2020neural}. \\
\underline{Chapter 2}: Amortizing statistical inference in exponential family models of neural activity \cite{bittner2019approximating}. \\
\underline{Chapter 3}: Inference in theoretical models of neural computation \cite{bittner2019interrogating}. \\

In our meeting, I'll summarize the latest advancements to the EPI project, and describe my contributions to the work listed in the thesis outline.

\bibliography{tcm}
\bibliographystyle{unsrt}
\vspace{3mm}

\end{document}

